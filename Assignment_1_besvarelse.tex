% Options for packages loaded elsewhere
\PassOptionsToPackage{unicode}{hyperref}
\PassOptionsToPackage{hyphens}{url}
%
\documentclass[
]{article}
\usepackage{amsmath,amssymb}
\usepackage{lmodern}
\usepackage{ifxetex,ifluatex}
\ifnum 0\ifxetex 1\fi\ifluatex 1\fi=0 % if pdftex
  \usepackage[T1]{fontenc}
  \usepackage[utf8]{inputenc}
  \usepackage{textcomp} % provide euro and other symbols
\else % if luatex or xetex
  \usepackage{unicode-math}
  \defaultfontfeatures{Scale=MatchLowercase}
  \defaultfontfeatures[\rmfamily]{Ligatures=TeX,Scale=1}
\fi
% Use upquote if available, for straight quotes in verbatim environments
\IfFileExists{upquote.sty}{\usepackage{upquote}}{}
\IfFileExists{microtype.sty}{% use microtype if available
  \usepackage[]{microtype}
  \UseMicrotypeSet[protrusion]{basicmath} % disable protrusion for tt fonts
}{}
\makeatletter
\@ifundefined{KOMAClassName}{% if non-KOMA class
  \IfFileExists{parskip.sty}{%
    \usepackage{parskip}
  }{% else
    \setlength{\parindent}{0pt}
    \setlength{\parskip}{6pt plus 2pt minus 1pt}}
}{% if KOMA class
  \KOMAoptions{parskip=half}}
\makeatother
\usepackage{xcolor}
\IfFileExists{xurl.sty}{\usepackage{xurl}}{} % add URL line breaks if available
\IfFileExists{bookmark.sty}{\usepackage{bookmark}}{\usepackage{hyperref}}
\hypersetup{
  pdftitle={Can R Notebooks help with reproducibility?},
  pdfauthor={Assignment 1 - MSB 105 - Ole Alexander Bakkevik \& Sindre Espedal},
  hidelinks,
  pdfcreator={LaTeX via pandoc}}
\urlstyle{same} % disable monospaced font for URLs
\usepackage[margin=1in]{geometry}
\usepackage{color}
\usepackage{fancyvrb}
\newcommand{\VerbBar}{|}
\newcommand{\VERB}{\Verb[commandchars=\\\{\}]}
\DefineVerbatimEnvironment{Highlighting}{Verbatim}{commandchars=\\\{\}}
% Add ',fontsize=\small' for more characters per line
\usepackage{framed}
\definecolor{shadecolor}{RGB}{248,248,248}
\newenvironment{Shaded}{\begin{snugshade}}{\end{snugshade}}
\newcommand{\AlertTok}[1]{\textcolor[rgb]{0.94,0.16,0.16}{#1}}
\newcommand{\AnnotationTok}[1]{\textcolor[rgb]{0.56,0.35,0.01}{\textbf{\textit{#1}}}}
\newcommand{\AttributeTok}[1]{\textcolor[rgb]{0.77,0.63,0.00}{#1}}
\newcommand{\BaseNTok}[1]{\textcolor[rgb]{0.00,0.00,0.81}{#1}}
\newcommand{\BuiltInTok}[1]{#1}
\newcommand{\CharTok}[1]{\textcolor[rgb]{0.31,0.60,0.02}{#1}}
\newcommand{\CommentTok}[1]{\textcolor[rgb]{0.56,0.35,0.01}{\textit{#1}}}
\newcommand{\CommentVarTok}[1]{\textcolor[rgb]{0.56,0.35,0.01}{\textbf{\textit{#1}}}}
\newcommand{\ConstantTok}[1]{\textcolor[rgb]{0.00,0.00,0.00}{#1}}
\newcommand{\ControlFlowTok}[1]{\textcolor[rgb]{0.13,0.29,0.53}{\textbf{#1}}}
\newcommand{\DataTypeTok}[1]{\textcolor[rgb]{0.13,0.29,0.53}{#1}}
\newcommand{\DecValTok}[1]{\textcolor[rgb]{0.00,0.00,0.81}{#1}}
\newcommand{\DocumentationTok}[1]{\textcolor[rgb]{0.56,0.35,0.01}{\textbf{\textit{#1}}}}
\newcommand{\ErrorTok}[1]{\textcolor[rgb]{0.64,0.00,0.00}{\textbf{#1}}}
\newcommand{\ExtensionTok}[1]{#1}
\newcommand{\FloatTok}[1]{\textcolor[rgb]{0.00,0.00,0.81}{#1}}
\newcommand{\FunctionTok}[1]{\textcolor[rgb]{0.00,0.00,0.00}{#1}}
\newcommand{\ImportTok}[1]{#1}
\newcommand{\InformationTok}[1]{\textcolor[rgb]{0.56,0.35,0.01}{\textbf{\textit{#1}}}}
\newcommand{\KeywordTok}[1]{\textcolor[rgb]{0.13,0.29,0.53}{\textbf{#1}}}
\newcommand{\NormalTok}[1]{#1}
\newcommand{\OperatorTok}[1]{\textcolor[rgb]{0.81,0.36,0.00}{\textbf{#1}}}
\newcommand{\OtherTok}[1]{\textcolor[rgb]{0.56,0.35,0.01}{#1}}
\newcommand{\PreprocessorTok}[1]{\textcolor[rgb]{0.56,0.35,0.01}{\textit{#1}}}
\newcommand{\RegionMarkerTok}[1]{#1}
\newcommand{\SpecialCharTok}[1]{\textcolor[rgb]{0.00,0.00,0.00}{#1}}
\newcommand{\SpecialStringTok}[1]{\textcolor[rgb]{0.31,0.60,0.02}{#1}}
\newcommand{\StringTok}[1]{\textcolor[rgb]{0.31,0.60,0.02}{#1}}
\newcommand{\VariableTok}[1]{\textcolor[rgb]{0.00,0.00,0.00}{#1}}
\newcommand{\VerbatimStringTok}[1]{\textcolor[rgb]{0.31,0.60,0.02}{#1}}
\newcommand{\WarningTok}[1]{\textcolor[rgb]{0.56,0.35,0.01}{\textbf{\textit{#1}}}}
\usepackage{graphicx}
\makeatletter
\def\maxwidth{\ifdim\Gin@nat@width>\linewidth\linewidth\else\Gin@nat@width\fi}
\def\maxheight{\ifdim\Gin@nat@height>\textheight\textheight\else\Gin@nat@height\fi}
\makeatother
% Scale images if necessary, so that they will not overflow the page
% margins by default, and it is still possible to overwrite the defaults
% using explicit options in \includegraphics[width, height, ...]{}
\setkeys{Gin}{width=\maxwidth,height=\maxheight,keepaspectratio}
% Set default figure placement to htbp
\makeatletter
\def\fps@figure{htbp}
\makeatother
\setlength{\emergencystretch}{3em} % prevent overfull lines
\providecommand{\tightlist}{%
  \setlength{\itemsep}{0pt}\setlength{\parskip}{0pt}}
\setcounter{secnumdepth}{-\maxdimen} % remove section numbering
\ifluatex
  \usepackage{selnolig}  % disable illegal ligatures
\fi
\newlength{\cslhangindent}
\setlength{\cslhangindent}{1.5em}
\newlength{\csllabelwidth}
\setlength{\csllabelwidth}{3em}
\newenvironment{CSLReferences}[2] % #1 hanging-ident, #2 entry spacing
 {% don't indent paragraphs
  \setlength{\parindent}{0pt}
  % turn on hanging indent if param 1 is 1
  \ifodd #1 \everypar{\setlength{\hangindent}{\cslhangindent}}\ignorespaces\fi
  % set entry spacing
  \ifnum #2 > 0
  \setlength{\parskip}{#2\baselineskip}
  \fi
 }%
 {}
\usepackage{calc}
\newcommand{\CSLBlock}[1]{#1\hfill\break}
\newcommand{\CSLLeftMargin}[1]{\parbox[t]{\csllabelwidth}{#1}}
\newcommand{\CSLRightInline}[1]{\parbox[t]{\linewidth - \csllabelwidth}{#1}\break}
\newcommand{\CSLIndent}[1]{\hspace{\cslhangindent}#1}

\title{Can R Notebooks help with reproducibility?}
\author{Assignment 1 - MSB 105 - Ole Alexander Bakkevik \& Sindre
Espedal}
\date{}

\begin{document}
\maketitle

\hypertarget{introduction}{%
\section{Introduction}\label{introduction}}

One might argue that there are two basic reasons to be concerned about
making research reproducible.

\emph{The first} is to show evidence of the correctness of your results.
Descriptions contained in scholarly publications are rarely sufficient
to convince skeptical readers of the reliability of our work. In simpler
times, scholarly publications showed the reader most of the work
involved in getting the result. The reader could make an informed choice
about the credibility of the science. Now, the reader may feel they are
being asked to blindly trust in all the details that were not described
in the original journal article.

Adopting a reproducible workflow means providing our audience with the
code and data that demonstrates the decisions we made as we generated
our results. This makes it easier for others to satisfy themselves that
our results are reliable (or not, since reproducibility is no guarantee
of correctness).

\emph{The second} reason to aspire to reproducibility is to enable
others to make use of methods and results. Equipped with only our
published article, our colleagues might struggle to reconstruct our
method in enough detail to apply it to their own data. Adopting a
reproducible workflow means publishing our code and data in order to
allow scientists to extend our approach to new applications with a
minimum of effort. This has the potential to save a great deal of time
in transmitting knowledge to future researchers.
\protect\hyperlink{ref-Git-reproducabilty}{\emph{Reproducibility Guide}}
(\protect\hyperlink{ref-Git-reproducabilty}{n.d.})

In this paper we will discuss the topics mentioned above.

\hypertarget{literature-review}{%
\section{Literature review}\label{literature-review}}

\hypertarget{reproducibility-r-notebooks}{%
\subsection{Reproducibility, R
notebooks}\label{reproducibility-r-notebooks}}

\protect\hyperlink{ref-peng2011}{Peng}
(\protect\hyperlink{ref-peng2011}{2011}) states that ``The standard of
reproducibility calls for the data and the computer code used to analyze
the data be made available to others'' . As a standard , this creates a
tedious and non-effective approach to replication. A far more beneficial
process is to independently inspect utilized data variables. R-notebooks
and other reproducible systems could serve as a crucial component in
verifying scientific results.

\hypertarget{replicability}{%
\subsection{Replicability}\label{replicability}}

Being able to replicate research results by other researchers is one
important part of the methodology in science. In the past, there has
been little testing of replicability. Reasons for this are that it is
not promoted to replicate another researcher's work. Criticism can also
arise about lack of creativity and imagination. A critical question is
also asked to the integrity of the researcher as one can be interpreted
as critical to the findings or that one does not trust the researcher.
Such arguments makes it less attractive to conduct replication studies.

\protect\hyperlink{ref-dewald1986}{Dewald et al.}
(\protect\hyperlink{ref-dewald1986}{1986}) tried to replicate a number
of data-sets and they found that accidental errors in empirical articles
are rather more common than unusual. Although it is quite common for
errors to occur in empirical economic research, it is quite frustrating
and difficult to replicate and build on the research when there are many
errors in the data-sets. This does not appear to significantly affect
the conclusion of the authors.

In recent times, technology has made it easier, cheaper and more
efficient to make and maintain journal archives. Still
\protect\hyperlink{ref-mccullough2008}{McCullough et al.}
(\protect\hyperlink{ref-mccullough2008}{2008}) finds that the potential
offered is reduced when editors fail to enforce and authors do not
adhere to the guidelines of the journal archives. It is noted that few
researchers use the opportunity as offered to engage in replication
because economic profession is considering replication as an ideal ``to
be known but not to be practiced''
(\protect\hyperlink{ref-mccullough2008}{McCullough et al., 2008}).

\hypertarget{possible-solutions}{%
\section{Possible solutions}\label{possible-solutions}}

\hypertarget{compendium-and-code-chunks}{%
\subsection{Compendium and ``Code
Chunks''}\label{compendium-and-code-chunks}}

\protect\hyperlink{ref-gentleman2007}{Gentleman and Lang}
(\protect\hyperlink{ref-gentleman2007}{2007}) points out that a
computable compendium might be an important tool for integrating codes
and data etc. This is because when such tools are collected and
assembled it must be possible to distribute and update, given that the
compendium is of the right quality, so will the possibility of
reproduction be simple.

Another possible solution is ``Code Chunks'' or ``Text Chunks.'' Code
and text chunks are a tool used to display data and code for
illustrations. Text chunks are used to describe and interpret results
and codes. Dynamic document will hence be an optimal compendium since
all the data and components will be available for reproduction
(\protect\hyperlink{ref-gentleman2007}{Gentleman and Lang, 2007}).

\hypertarget{incentivizing-reproducibility}{%
\subsection{Incentivizing
Reproducibility}\label{incentivizing-reproducibility}}

Over the past several years, a series of publications and policy
statements have generated increasing awareness in the scientific
community of the scale and implications of the problem of irreproducible
data---or at least lack of robust results---particularly in the realm of
basic and translational research.

Recent studies have shown that the key findings in 50\% or more of
published reports in certain fields cannot be reproduced. As the public,
government, and private funders of research comprehend the extent of the
problem, trust in the scientific enterprise erodes, and confidence in
the ability of the scientific community to address this problem wanes.
In addition, there is considerable potential for reputational damage to
scientists, universities, and entire fields (for example, cancer
biology, genomics, and psychology).
\protect\hyperlink{ref-Science.org}{\emph{An Incentive-Based Approach
for Improving Data Reproducibility}}
(\protect\hyperlink{ref-Science.org}{n.d.})

One possible cause of irreproducible-data is stated by Hessen
as``\emph{Scientists are incentivized to produce more results at the
expense of spending more time on the reproducibility of any given
result}.'' Hessen furthermore list three possible solutions:

\begin{itemize}
\item
  One solution is to eliminate imperfections in the peer review
  system.\\
  \emph{(Without those imperfections credit incentives are perfectly
  aligned with the social optimum in Hessen`s model)}
\item
  Another solution focuses on the amount of credit given for
  irreproducible results.\\
  \emph{(If the credit given to irreproducible results matched the
  social value of those results more closely, the gap between the
  credit-maximizing optimum and the social optimum would be reduced)}
\item
  A third solution aims to compensate for the misalignment.\\
  (l\emph{imiting the number of papers scientists may publish per unit
  time) \protect\hyperlink{ref-schulz2016}{Schulz et al.}
  (\protect\hyperlink{ref-schulz2016}{2016})}
\end{itemize}

\hypertarget{incentivizing-gone-wrong}{%
\subsection{Incentivizing gone wrong}\label{incentivizing-gone-wrong}}

A good example of fraudulent science is Andrew Wakefield and his study
on the link between autism and the MMR vaccine published in the Lancet.
Wakefield was paid by a Legal Aid Board of parents of children with
autism to conduct a pilot study of virological investigation in autistic
children, some of whom were included in the Lancet publication.
Additionally, Wakefield most likely manipulated the data, thus
presenting false results. Since then Wakefield has become the
``\emph{godfather}'' for the anti-vaccine movement, a movement whom have
grown exponentially during the covid-19 pandemic.
\protect\hyperlink{ref-schulz2016}{Schulz et al.}
(\protect\hyperlink{ref-schulz2016}{2016})

\hypertarget{example-list-2-level}{%
\subsection{Example list 2 level}\label{example-list-2-level}}

Example of a \emph{code chunk} in an R Notebook:

\begin{Shaded}
\begin{Highlighting}[]
\NormalTok{l }\OtherTok{\textless{}{-}} \FunctionTok{list}\NormalTok{(}\AttributeTok{x =} \DecValTok{1}\SpecialCharTok{:}\DecValTok{4}\NormalTok{, }\AttributeTok{y =} \FunctionTok{c}\NormalTok{(}\ConstantTok{TRUE}\NormalTok{, }\ConstantTok{FALSE}\NormalTok{, }\ConstantTok{FALSE}\NormalTok{), }\AttributeTok{z =} \FunctionTok{c}\NormalTok{(}\StringTok{"aa"}\NormalTok{, }\StringTok{"bb"}\NormalTok{), }\AttributeTok{zz=} \FunctionTok{c}\NormalTok{(}\FloatTok{2.1}\NormalTok{, }\FloatTok{4.33}\NormalTok{))}
\FunctionTok{str}\NormalTok{(l)}
\end{Highlighting}
\end{Shaded}

\begin{verbatim}
## List of 4
##  $ x : int [1:4] 1 2 3 4
##  $ y : logi [1:3] TRUE FALSE FALSE
##  $ z : chr [1:2] "aa" "bb"
##  $ zz: num [1:2] 2.1 4.33
\end{verbatim}

\hypertarget{session-info}{%
\subsection{Session info}\label{session-info}}

Example session info:

\begin{Shaded}
\begin{Highlighting}[]
\FunctionTok{sessionInfo}\NormalTok{()}
\end{Highlighting}
\end{Shaded}

\begin{verbatim}
## R version 4.1.1 (2021-08-10)
## Platform: x86_64-apple-darwin17.0 (64-bit)
## Running under: macOS Catalina 10.15.7
## 
## Matrix products: default
## BLAS:   /Library/Frameworks/R.framework/Versions/4.1/Resources/lib/libRblas.0.dylib
## LAPACK: /Library/Frameworks/R.framework/Versions/4.1/Resources/lib/libRlapack.dylib
## 
## locale:
## [1] en_US.UTF-8/en_US.UTF-8/en_US.UTF-8/C/en_US.UTF-8/en_US.UTF-8
## 
## attached base packages:
## [1] stats     graphics  grDevices utils     datasets  methods   base     
## 
## loaded via a namespace (and not attached):
##  [1] compiler_4.1.1    magrittr_2.0.1    tools_4.1.1       htmltools_0.5.1.1
##  [5] yaml_2.2.1        stringi_1.7.3     rmarkdown_2.10    knitr_1.33       
##  [9] stringr_1.4.0     xfun_0.25         digest_0.6.27     rlang_0.4.11     
## [13] evaluate_0.14
\end{verbatim}

The session info function provides the reader information regarding
which operating system, packages and data sets that have been used. This
information is crucial in terms of gaining reproducibility.

\hypertarget{reproducibility-across-sectors}{%
\subsection{Reproducibility across
sectors}\label{reproducibility-across-sectors}}

Other areas where application of reproducibility would prove beneficiary
is e.g.~the pharmaceutical industry. Present day studies show that
replicating present day clinical-research data is demanding. Which often
leads to drugs having to prolong their release to actual patient trials.
One human factor could be the fear of being ``discredited'' among peers,
which lead to an bias among researchers. Ultimately causing studies not
to be reproduced. \protect\hyperlink{ref-Pharm-tech}{\emph{Why Is
Reproducing Pharmaceutical Medical Research so Hard?}}
(\protect\hyperlink{ref-Pharm-tech}{n.d.})

\hypertarget{conclusion}{%
\section{Conclusion}\label{conclusion}}

Providing studies that are reproducible is vital in terms of quality
assurance and cost- effectiveness. In addition deterring fraudulent
scientists is crucial.

By using dynamic documents in the form of codes, data, explanations,
etc. in the form of code chunks and text chunks, there are good
opportunities for both replication and reproducibility of research, and
also further research on previous studies.

\begin{itemize}
\tightlist
\item
  Motivate researchers to share to make their work available
\item
  Disadvantage? maybe too many different packages (difficult to keep
  track)
\end{itemize}

\newpage

\hypertarget{references}{%
\section{References}\label{references}}

\hypertarget{refs}{}
\begin{CSLReferences}{1}{0}
\leavevmode\hypertarget{ref-Science.org}{}%
\emph{An incentive-based approach for improving data reproducibility}.
(n.d.).
\url{https://www.science.org/doi/full/10.1126/scitranslmed.aaf5003}

\leavevmode\hypertarget{ref-dewald1986}{}%
Dewald, W. G., Thursby, J. G., and Anderson, R. G. (1986). Replication
in {Empirical Economics}: {The Journal} of {Money}, {Credit} and
{Banking Project}. \emph{The American Economic Review}, \emph{76}(4),
587--603.

\leavevmode\hypertarget{ref-gentleman2007}{}%
Gentleman, R., and Lang, D. T. (2007). Statistical analyses and
reproducible research. \emph{Journal of Computational and Graphical
Statistics}, \emph{16}(1), 1--23.
\url{https://doi.org/10.1198/106186007X178663}

\leavevmode\hypertarget{ref-mccullough2008}{}%
McCullough, B. D., McGeary, K. A., and Harrison, T. D. (2008). Do
economics journal archives promote replicable research? \emph{Canadian
Journal of Economics/Revue Canadienne d'économique}, \emph{41}(4),
1406--1420. \url{https://doi.org/10.1111/j.1540-5982.2008.00509.x}

\leavevmode\hypertarget{ref-peng2011}{}%
Peng, R. D. (2011). Reproducible {Research} in {Computational Science}.
\emph{Science}, \emph{334}(6060), 1226--1227.
\url{https://doi.org/10.1126/science.1213847}

\leavevmode\hypertarget{ref-Git-reproducabilty}{}%
\emph{Reproducibility guide}. (n.d.).
\url{https://ropensci.github.io/reproducibility-guide/sections/introduction/}

\leavevmode\hypertarget{ref-schulz2016}{}%
Schulz, J. B., Cookson, M. R., and Hausmann, L. (2016). The impact of
fraudulent and irreproducible data to the translational research crisis
{{}} solutions and implementation. \emph{Journal of Neurochemistry},
\emph{139}(S2), 253--270. \url{https://doi.org/10.1111/jnc.13844}

\leavevmode\hypertarget{ref-Pharm-tech}{}%
\emph{Why is reproducing pharmaceutical medical research so hard?}
(n.d.).
\url{https://www.pharmaceutical-technology.com/features/why-is-it-so-hard-to-reproduce-medical-research-results/}

\end{CSLReferences}

\hypertarget{appendix}{%
\section{Appendix}\label{appendix}}

\hypertarget{display-of-git-commits-and-three-branches}{%
\subsection{Display of Git commits and three
branches}\label{display-of-git-commits-and-three-branches}}

\includegraphics{images/paste-7A7BFE4C.png}

\end{document}
