% Options for packages loaded elsewhere
\PassOptionsToPackage{unicode}{hyperref}
\PassOptionsToPackage{hyphens}{url}
%
\documentclass[
  12pt,
  norsk,
]{article}
\usepackage{amsmath,amssymb}
\usepackage{lmodern}
\usepackage{ifxetex,ifluatex}
\ifnum 0\ifxetex 1\fi\ifluatex 1\fi=0 % if pdftex
  \usepackage[T1]{fontenc}
  \usepackage[utf8]{inputenc}
  \usepackage{textcomp} % provide euro and other symbols
\else % if luatex or xetex
  \usepackage{unicode-math}
  \defaultfontfeatures{Scale=MatchLowercase}
  \defaultfontfeatures[\rmfamily]{Ligatures=TeX,Scale=1}
\fi
% Use upquote if available, for straight quotes in verbatim environments
\IfFileExists{upquote.sty}{\usepackage{upquote}}{}
\IfFileExists{microtype.sty}{% use microtype if available
  \usepackage[]{microtype}
  \UseMicrotypeSet[protrusion]{basicmath} % disable protrusion for tt fonts
}{}
\makeatletter
\@ifundefined{KOMAClassName}{% if non-KOMA class
  \IfFileExists{parskip.sty}{%
    \usepackage{parskip}
  }{% else
    \setlength{\parindent}{0pt}
    \setlength{\parskip}{6pt plus 2pt minus 1pt}}
}{% if KOMA class
  \KOMAoptions{parskip=half}}
\makeatother
\usepackage{xcolor}
\IfFileExists{xurl.sty}{\usepackage{xurl}}{} % add URL line breaks if available
\IfFileExists{bookmark.sty}{\usepackage{bookmark}}{\usepackage{hyperref}}
\hypersetup{
  pdftitle={«R Notebooks» og reproduserbarhet},
  pdfauthor={Assignment 1 i kurset Data Science 2020},
  pdflang={nb-NO},
  hidelinks,
  pdfcreator={LaTeX via pandoc}}
\urlstyle{same} % disable monospaced font for URLs
\usepackage[margin=1in]{geometry}
\usepackage{color}
\usepackage{fancyvrb}
\newcommand{\VerbBar}{|}
\newcommand{\VERB}{\Verb[commandchars=\\\{\}]}
\DefineVerbatimEnvironment{Highlighting}{Verbatim}{commandchars=\\\{\}}
% Add ',fontsize=\small' for more characters per line
\usepackage{framed}
\definecolor{shadecolor}{RGB}{248,248,248}
\newenvironment{Shaded}{\begin{snugshade}}{\end{snugshade}}
\newcommand{\AlertTok}[1]{\textcolor[rgb]{0.94,0.16,0.16}{#1}}
\newcommand{\AnnotationTok}[1]{\textcolor[rgb]{0.56,0.35,0.01}{\textbf{\textit{#1}}}}
\newcommand{\AttributeTok}[1]{\textcolor[rgb]{0.77,0.63,0.00}{#1}}
\newcommand{\BaseNTok}[1]{\textcolor[rgb]{0.00,0.00,0.81}{#1}}
\newcommand{\BuiltInTok}[1]{#1}
\newcommand{\CharTok}[1]{\textcolor[rgb]{0.31,0.60,0.02}{#1}}
\newcommand{\CommentTok}[1]{\textcolor[rgb]{0.56,0.35,0.01}{\textit{#1}}}
\newcommand{\CommentVarTok}[1]{\textcolor[rgb]{0.56,0.35,0.01}{\textbf{\textit{#1}}}}
\newcommand{\ConstantTok}[1]{\textcolor[rgb]{0.00,0.00,0.00}{#1}}
\newcommand{\ControlFlowTok}[1]{\textcolor[rgb]{0.13,0.29,0.53}{\textbf{#1}}}
\newcommand{\DataTypeTok}[1]{\textcolor[rgb]{0.13,0.29,0.53}{#1}}
\newcommand{\DecValTok}[1]{\textcolor[rgb]{0.00,0.00,0.81}{#1}}
\newcommand{\DocumentationTok}[1]{\textcolor[rgb]{0.56,0.35,0.01}{\textbf{\textit{#1}}}}
\newcommand{\ErrorTok}[1]{\textcolor[rgb]{0.64,0.00,0.00}{\textbf{#1}}}
\newcommand{\ExtensionTok}[1]{#1}
\newcommand{\FloatTok}[1]{\textcolor[rgb]{0.00,0.00,0.81}{#1}}
\newcommand{\FunctionTok}[1]{\textcolor[rgb]{0.00,0.00,0.00}{#1}}
\newcommand{\ImportTok}[1]{#1}
\newcommand{\InformationTok}[1]{\textcolor[rgb]{0.56,0.35,0.01}{\textbf{\textit{#1}}}}
\newcommand{\KeywordTok}[1]{\textcolor[rgb]{0.13,0.29,0.53}{\textbf{#1}}}
\newcommand{\NormalTok}[1]{#1}
\newcommand{\OperatorTok}[1]{\textcolor[rgb]{0.81,0.36,0.00}{\textbf{#1}}}
\newcommand{\OtherTok}[1]{\textcolor[rgb]{0.56,0.35,0.01}{#1}}
\newcommand{\PreprocessorTok}[1]{\textcolor[rgb]{0.56,0.35,0.01}{\textit{#1}}}
\newcommand{\RegionMarkerTok}[1]{#1}
\newcommand{\SpecialCharTok}[1]{\textcolor[rgb]{0.00,0.00,0.00}{#1}}
\newcommand{\SpecialStringTok}[1]{\textcolor[rgb]{0.31,0.60,0.02}{#1}}
\newcommand{\StringTok}[1]{\textcolor[rgb]{0.31,0.60,0.02}{#1}}
\newcommand{\VariableTok}[1]{\textcolor[rgb]{0.00,0.00,0.00}{#1}}
\newcommand{\VerbatimStringTok}[1]{\textcolor[rgb]{0.31,0.60,0.02}{#1}}
\newcommand{\WarningTok}[1]{\textcolor[rgb]{0.56,0.35,0.01}{\textbf{\textit{#1}}}}
\usepackage{graphicx}
\makeatletter
\def\maxwidth{\ifdim\Gin@nat@width>\linewidth\linewidth\else\Gin@nat@width\fi}
\def\maxheight{\ifdim\Gin@nat@height>\textheight\textheight\else\Gin@nat@height\fi}
\makeatother
% Scale images if necessary, so that they will not overflow the page
% margins by default, and it is still possible to overwrite the defaults
% using explicit options in \includegraphics[width, height, ...]{}
\setkeys{Gin}{width=\maxwidth,height=\maxheight,keepaspectratio}
% Set default figure placement to htbp
\makeatletter
\def\fps@figure{htbp}
\makeatother
\setlength{\emergencystretch}{3em} % prevent overfull lines
\providecommand{\tightlist}{%
  \setlength{\itemsep}{0pt}\setlength{\parskip}{0pt}}
\setcounter{secnumdepth}{-\maxdimen} % remove section numbering
\ifxetex
  % Load polyglossia as late as possible: uses bidi with RTL langages (e.g. Hebrew, Arabic)
  \usepackage{polyglossia}
  \setmainlanguage[]{norsk}
\else
  \usepackage[main=norsk]{babel}
% get rid of language-specific shorthands (see #6817):
\let\LanguageShortHands\languageshorthands
\def\languageshorthands#1{}
\fi
\ifluatex
  \usepackage{selnolig}  % disable illegal ligatures
\fi
\newlength{\cslhangindent}
\setlength{\cslhangindent}{1.5em}
\newlength{\csllabelwidth}
\setlength{\csllabelwidth}{3em}
\newenvironment{CSLReferences}[2] % #1 hanging-ident, #2 entry spacing
 {% don't indent paragraphs
  \setlength{\parindent}{0pt}
  % turn on hanging indent if param 1 is 1
  \ifodd #1 \everypar{\setlength{\hangindent}{\cslhangindent}}\ignorespaces\fi
  % set entry spacing
  \ifnum #2 > 0
  \setlength{\parskip}{#2\baselineskip}
  \fi
 }%
 {}
\usepackage{calc}
\newcommand{\CSLBlock}[1]{#1\hfill\break}
\newcommand{\CSLLeftMargin}[1]{\parbox[t]{\csllabelwidth}{#1}}
\newcommand{\CSLRightInline}[1]{\parbox[t]{\linewidth - \csllabelwidth}{#1}\break}
\newcommand{\CSLIndent}[1]{\hspace{\cslhangindent}#1}

\title{«R Notebooks» og reproduserbarhet}
\author{Assignment 1 i kurset Data Science 2020}
\date{}

\begin{document}
\maketitle

Skriv et kort notat --- 5-7 sider (inklusive appendiks) i form av en «R
Notebook» --- der nødvendigheten av reproduserbarhet i forskning
diskuteres. Diskuter også om bruk av «R Notebooks» er en mulig løsning
på problemet med manglende reproduserbarhet.

Dokumentet må inneholde følgende bruk av R markdown:

\begin{enumerate}
\def\labelenumi{\arabic{enumi})}
\tightlist
\item
  Minst 4 overskrifter
\item
  Minst 1 ordnet liste på 2 nivå
\item
  Minst 1 eksempel på bruk av

  \begin{enumerate}
  \def\labelenumii{\arabic{enumii})}
  \tightlist
  \item
    \textbf{halv-fet skrift(«bold»)},
  \item
    \emph{kursiv skrift («italic»)} og
  \item
    \textbf{\emph{halv-fet kursiv skrift}}
  \end{enumerate}
\item
  Minst 1 internt bilde skal være screenshot av git history som:

  \begin{enumerate}
  \def\labelenumii{\arabic{enumii})}
  \tightlist
  \item
    Dokumenterer minst 10 «commits»
  \item
    Dokumenterer bruk av minst 3 «branches»
  \item
    Ekstra stjerne til dem som klarer å få til en «merge conflict» ;-)
  \item
    Bildet som dokumenterer git history skal være i et appendiks som
    kommer helt til slutt i dokumentet (etter referansene)
  \end{enumerate}
\item
  Kjør \texttt{sessionInfo()} i en code-chunk (husk å gi chunk-en navn).
  Hvordan kan denne funksjonen hjelpe oss med å gjøre et dokument
  reproduserbart?
\item
  Vi benytter apa for sitering og referanseliste
  (\texttt{apa-no-ampersand.csl} er tilgjengelig under \emph{Filer} i
  Canvas.)
\item
  Bruk begge siteringsformene, dvs med og uten \texttt{{[}{]}}

  \begin{enumerate}
  \def\labelenumii{\arabic{enumii})}
  \tightlist
  \item
    Husk at for å få siteringsinfo for R pakker kan dere bruke
    kommandoen
    \texttt{toBibtex(citation(\textless{}navn-R-pakke\textgreater{}))} ,
    f.eks
  \end{enumerate}
\end{enumerate}

\begin{Shaded}
\begin{Highlighting}[]
\FunctionTok{toBibtex}\NormalTok{(}\FunctionTok{citation}\NormalTok{(}\StringTok{"rmarkdown"}\NormalTok{))}
\end{Highlighting}
\end{Shaded}

\begin{verbatim}
## @Manual{,
##   title = {rmarkdown: Dynamic Documents for R},
##   author = {JJ Allaire and Yihui Xie and Jonathan McPherson and Javier Luraschi and Kevin Ushey and Aron Atkins and Hadley Wickham and Joe Cheng and Winston Chang and Richard Iannone},
##   year = {2021},
##   note = {R package version 2.10},
##   url = {https://github.com/rstudio/rmarkdown},
## }
## 
## @Book{,
##   title = {R Markdown: The Definitive Guide},
##   author = {Yihui Xie and J.J. Allaire and Garrett Grolemund},
##   publisher = {Chapman and Hall/CRC},
##   address = {Boca Raton, Florida},
##   year = {2018},
##   note = {ISBN 9781138359338},
##   url = {https://bookdown.org/yihui/rmarkdown},
## }
## 
## @Book{,
##   title = {R Markdown Cookbook},
##   author = {Yihui Xie and Christophe Dervieux and Emily Riederer},
##   publisher = {Chapman and Hall/CRC},
##   address = {Boca Raton, Florida},
##   year = {2020},
##   note = {ISBN 9780367563837},
##   url = {https://bookdown.org/yihui/rmarkdown-cookbook},
## }
\end{verbatim}

Velg en «entry» --- f.eks. fra \texttt{@Manual\{,} t.o.m. \texttt{\}}
--- vha. musen og kopier denne valgte teksten. Gå så inn i Zotero og
velg \texttt{Importer\ fra\ utklippstavle} fra \texttt{Fil} menyen.

\hypertarget{forslag-til-litteratur}{%
\subsection{Forslag til litteratur}\label{forslag-til-litteratur}}

Se foredrag som ligger under \emph{Kursets mediefiler} på Canvas.
bib-filen som er brukt for referansene i foredrag er også lagt ut på
Canvas (under \emph{Filer}).

For generelle tanker rundt reproduserbarhet er
\protect\hyperlink{ref-peng2011}{Peng}
(\protect\hyperlink{ref-peng2011}{2011}) en god kilde. Videre gir
\protect\hyperlink{ref-mccullough2008}{McCullough et al.}
(\protect\hyperlink{ref-mccullough2008}{2008}) en god illustrasjon av
problemets omfang innen fagområdet økonomi.
\protect\hyperlink{ref-mccullough2008}{McCullough et al.}
(\protect\hyperlink{ref-mccullough2008}{2008}) diskuterer også om
tidsskriftenes arkiver av datasett og programkode er en
tilfredsstillende løsning av problemet.

Basert på tanker fra \protect\hyperlink{ref-knuth1992}{Knuth}
(\protect\hyperlink{ref-knuth1992}{1992}) introduserte
(\protect\hyperlink{ref-gentleman2004}{\textbf{gentleman2004?}})
begrepet «compendium» som:

\begin{quote}
both a container for the different elements that make up the document
and its computations (i.e.~text, code, data, \ldots), and as a means for
distributing, managing and updating the collection.
\end{quote}

Dokumentet nevnt ovenfor er det
(\protect\hyperlink{ref-gentleman2004}{\textbf{gentleman2004?}})\footnote{Robert
  Gentleman er sammen med Ross Ihaka regnet som «fedrene» til R} omtaler
som «dynamic documents». Artikkelen drøfter også disse to begrepenes
relevans for «reproducible research». Videre introduseres også «code
chunks»

\begin{quote}
sequences of commands in some programming language such as R or Perl.
Code chunks are intended to be evaluated according to the language in
which they are written. These perform the computations needed to produce
the appropriate output within the paper, and also to produce
intermediate results used across different code chunks.
\end{quote}

og «text chunks» som beskrives som:

\begin{quote}
Text chunks describe the problem, the code, the results and often their
interpretation. Text chunks are intended to be formatted for reading.
\end{quote}

Disse tankene er også blitt brukt til å gjenskape deler av
\protect\hyperlink{ref-golub1999}{Golub et al.}
(\protect\hyperlink{ref-golub1999}{1999}) som et slikt «compendium»
(\protect\hyperlink{ref-gentleman2005}{Gentleman, 2005}). Dette for å
vise at idéen er gjennomførbar i prakis.

Når det gjelder «R notebooks», som kanskje kan betraktes som en
implementering av et «compendium», er disse avhengige av de to pakkene
\texttt{rmarkdown}, (\protect\hyperlink{ref-allaire2020}{Allaire et al.,
2020}), og \texttt{knitr}, (\protect\hyperlink{ref-xie2020}{Xie, 2020}).
R markdown og tilhørende programvare er kanskje best beskrevet i
\protect\hyperlink{ref-xie2018}{Xie et al.}
(\protect\hyperlink{ref-xie2018}{2018}) og
\protect\hyperlink{ref-riederer}{Riederer}
(\protect\hyperlink{ref-riederer}{u.å.}) .

\hypertarget{et-forslag-til-disposisjon-dere-trenger-ikke-dekke-alt-listet-her}{%
\subsection{Et forslag til disposisjon (dere trenger ikke dekke alt
listet
her)}\label{et-forslag-til-disposisjon-dere-trenger-ikke-dekke-alt-listet-her}}

\begin{itemize}
\tightlist
\item
  Innleding

  \begin{itemize}
  \tightlist
  \item
    Reproduserbarhet, R notebooks
  \end{itemize}
\item
  Litteraturgjennomgang

  \begin{itemize}
  \tightlist
  \item
    Replikerbarhet/reproduserbarhet
  \item
    Problemets omfang

    \begin{itemize}
    \tightlist
    \item
      Vil dagens løsning med arkiv av data og event. programkode hos
      tidsskriftene kunne løse problemet?
    \end{itemize}
  \item
    Mulig løsning (teoretisk plan):

    \begin{itemize}
    \tightlist
    \item
      «Compendium», «Dynamic document», «code chunck» og «text chunk»
    \end{itemize}
  \item
    Mulig løsning:

    \begin{itemize}
    \tightlist
    \item
      R Notebooks
    \end{itemize}
  \end{itemize}
\item
  Analyse

  \begin{itemize}
  \tightlist
  \item
    Løser R notebooks problemet med reproduserbarhet

    \begin{itemize}
    \tightlist
    \item
      helt eller bare delvis
    \end{itemize}
  \item
    Eksempler på «code chunks» («R Code Block») og «text chunck» i R
    notebook
  \item
    Har forskerne incentiver til å være «reproduserbare», eller må de
    tvinges?
  \item
    Er økt reproduserbarhet noe som vil tvinge seg frem eller er dagens
    økte interesse bare et blaff?
  \item
    Kan reproduserbarhet ha relevans i sektorer utenfor akademia?
  \end{itemize}
\item
  Konklusjon
\item
  Litteraturliste
\end{itemize}

R Notebook-en må kunne transformeres til .nb.html, .pdf fil. og MS Word
format.

\hypertarget{litteraturliste}{%
\subsection{Litteraturliste}\label{litteraturliste}}

\hypertarget{refs}{}
\begin{CSLReferences}{1}{0}
\leavevmode\hypertarget{ref-allaire2020}{}%
Allaire, J., Xie, Y., McPherson, J., Luraschi, J., Ushey, K., Atkins,
A., Wickham, H., Cheng, J., Chang, W., og Iannone, R. (2020).
\emph{Rmarkdown: {Dynamic} Documents for r}.

\leavevmode\hypertarget{ref-gentleman2005}{}%
Gentleman, R. (2005). Reproducible {Research}: {A Bioinformatics Case
Study}. \emph{Statistical Applications in Genetics and Molecular
Biology}, \emph{4}(1). \url{https://doi.org/10.2202/1544-6115.1034}

\leavevmode\hypertarget{ref-golub1999}{}%
Golub, T. R., Slonim, D. K., Tamayo, P., Huard, C., Gaasenbeek, M.,
Mesirov, J. P., Coller, H., Loh, M. L., Downing, J. R., Caligiuri, M.
A., Bloomfield, C. D., og Lander, E. S. (1999). Molecular Classification
of Cancer: Class Discovery and Class Prediction by Gene Expression
Monitoring. \emph{Science (New York, N.Y.)}, \emph{286}(5439), 531--537.
\url{https://doi.org/10.1126/science.286.5439.531}

\leavevmode\hypertarget{ref-knuth1992}{}%
Knuth, D. E. (1992). \emph{Literate {Programming}}. {Cambridge
University Press}.

\leavevmode\hypertarget{ref-mccullough2008}{}%
McCullough, B. D., McGeary, K. A., og Harrison, T. D. (2008). Do
Economics Journal Archives Promote Replicable Research? \emph{Canadian
Journal of Economics/Revue canadienne d'économique}, \emph{41}(4),
1406--1420. \url{https://doi.org/10.1111/j.1540-5982.2008.00509.x}

\leavevmode\hypertarget{ref-peng2011}{}%
Peng, R. D. (2011). Reproducible {Research} in {Computational Science}.
\emph{Science}, \emph{334}(6060), 1226--1227.
\url{https://doi.org/10.1126/science.1213847}

\leavevmode\hypertarget{ref-riederer}{}%
Riederer, E., Christophe Dervieux. (u.å.). \emph{R {Markdown Cookbook}}.

\leavevmode\hypertarget{ref-xie2020}{}%
Xie, Y. (2020). \emph{Knitr: {A} General-Purpose Package for Dynamic
Report Generation in r} {[}Manual{]}.

\leavevmode\hypertarget{ref-xie2018}{}%
Xie, Y., Allaire, J. J., og Grolemund, G. (2018). \emph{R Markdown:
{The} Definitive Guide}. {Chapman and Hall/CRC}.

\end{CSLReferences}

\end{document}
